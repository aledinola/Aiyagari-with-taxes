%\usepackage{float}
% for subfigures
% relative path
%\graphicspath{{subdir1/}{subdir2/}{subdir3/}...{subdirn/}}
%% User-defined commands
%\usepackage[top=1in, bottom=1in, left=1in, right=1in]{geometry}     % adjust page margins


\documentclass[12pt]{article}
%%%%%%%%%%%%%%%%%%%%%%%%%%%%%%%%%%%%%%%%%%%%%%%%%%%%%%%%%%%%%%%%%%%%%%%%%%%%%%%%%%%%%%%%%%%%%%%%%%%%%%%%%%%%%%%%%%%%%%%%%%%%%%%%%%%%%%%%%%%%%%%%%%%%%%%%%%%%%%%%%%%%%%%%%%%%%%%%%%%%%%%%%%%%%%%%%%%%%%%%%%%%%%%%%%%%%%%%%%%%%%%%%%%%%%%%%%%%%%%%%%%%%%%%%%%%
\usepackage[latin9]{inputenc}
\usepackage[a4paper]{geometry}
\usepackage{color}
\usepackage{array}
\usepackage[capposition=top]{floatrow}
\usepackage{multirow}
\usepackage{amsmath}
\usepackage{amssymb}
\usepackage{graphicx}
\usepackage{esint}
\usepackage[authoryear]{natbib}
\usepackage{ulem}
\usepackage{arydshln}
\usepackage{rotating}
\usepackage[unicode=true,
bookmarks=true,bookmarksnumbered=false,bookmarksopen=true,bookmarksopenlevel=1,
breaklinks=true,pdfborder={0 0 1},backref=section,colorlinks=true]{hyperref}
\usepackage{booktabs}
\usepackage{amsthm}
\usepackage{color}
\usepackage{placeins}
\usepackage{subcaption}
\usepackage{amsfonts}
\usepackage[normalem]{ulem}
\usepackage[dvipsnames]{xcolor}
\usepackage{enumitem}

\setcounter{MaxMatrixCols}{10}
%TCIDATA{OutputFilter=Latex.dll}
%TCIDATA{Version=5.50.0.2953}
%TCIDATA{<META NAME="SaveForMode" CONTENT="1">}
%TCIDATA{BibliographyScheme=BibTeX}
%TCIDATA{LastRevised=Saturday, May 11, 2024 22:48:51}
%TCIDATA{<META NAME="GraphicsSave" CONTENT="32">}

\geometry{verbose}
\PassOptionsToPackage{normalem}{ulem}
\hypersetup{
plainpages=false,urlcolor=magenta,citecolor=magenta,linkcolor=blue,pdfstartview=FitH,pdfview=FitH,plainpages=false,urlcolor=blue,citecolor=blue,linkcolor=blue,pdfstartview=FitH,pdfview=FitH}
\makeatletter
\graphicspath{{Figures/}} 
\setlength{\marginparwidth}{0in} \setlength{\marginparsep}{0in}
\setlength{\oddsidemargin}{0in} \setlength{\evensidemargin}{0in}
\setlength{\textwidth}{6.35in} \setlength{\topmargin}{-.50in}
\setlength{\textheight}{9.45in}
\renewcommand{\baselinestretch}{1.2}\small\normalsize
\providecommand{\tabularnewline}{\\}
\newcommand{\ssskip}{\vspace*{0.05cm}}
\newcommand{\sskip}{\vspace*{0.15cm}}
\newcommand{\lskip}{\vspace*{0.45cm}}
\makeatother

\input{tcilatex}

\begin{document}

\title{\textbf{Aiyagari model with progressive taxes}}
\date{\today }
\author{}
\maketitle

\section{Model}

Here I present a brief description of the model. Households face the
following problem:%
\begin{equation*}
V\left( a,z\right) =\max_{c,a^{\prime }}\left\{ \frac{c^{1-\sigma }-1}{%
1-\sigma }+\beta \sum_{z^{\prime }}\Gamma _{z,z^{\prime }}V\left( a^{\prime
},z^{\prime }\right) \right\} 
\end{equation*}%
subject to%
\begin{equation*}
y=wz+ra
\end{equation*}%
\begin{equation*}
c+a^{\prime }=a+y-T(y)
\end{equation*}%
\begin{equation*}
c\geq 0,\ \ a^{\prime }\geq 0.
\end{equation*}%
The individual state variables are asset holdings $a$ (endogenous state
variable) and the idiosyncratic shock $z$ (exogenous state variable ). $%
\Gamma _{z,z^{\prime }}$ denotes the transition matrix of the Markov chain
over $z$, with $\sum_{z^{\prime }}\Gamma _{z,z^{\prime }}=1$ for all $z$.
Taxes are given by%
\begin{equation*}
T(y)=y-\lambda y^{1-\tau }.
\end{equation*}%
The parameter $\tau \in \lbrack 0,1)$ denotes tax progressivity. If $\tau =0$%
, then taxes are proportional to income, with an average tax rate equal to $%
\left( 1-\lambda \right) $. If instead $\tau \in \left( 0,1\right) $, the
average tax rate%
\begin{equation*}
\frac{T(y)}{y}=1-\lambda y^{-\tau }
\end{equation*}%
is increasing with respect to income, i.e. the tax system is progressive.
Households choose consumption $c$, and next-period assets $a^{\prime }$. The
policy function are denoted as $c=g_{c}\left( a,z\right) $, and $a^{\prime
}=g_{a}\left( a,z\right) $.

Factor prices $r$ and $w$ are pinned down by the first-order conditions of
the representative firm%
\begin{equation*}
r=\left( 1-\alpha \right) \left( \frac{K}{L}\right) ^{\alpha -1}-\delta 
\end{equation*}%
\begin{equation*}
w=\left( 1-\alpha \right) \left( \frac{K}{L}\right) ^{\alpha }
\end{equation*}%
and the aggregate production function is 
\begin{equation*}
Y=K^{\alpha }L^{1-\alpha }.
\end{equation*}

The market clearing conditions are%
\begin{equation*}
K=\int g_{a}\left( a,z\right) d\mu \left( a,z\right) 
\end{equation*}%
\begin{equation*}
L=\int zd\mu \left( a,z\right) ,
\end{equation*}%
where $\mu $ is the stationary distribution. The government budget
constraint is simply%
\begin{equation*}
G=\int T(wz+ra)d\mu (a,z)
\end{equation*}%
where $G$ denotes wasteful government spending. Then, by Walras' law, the
aggregate resource constraint of the economy is automatically satisfied:%
\begin{equation*}
C+\delta K+G=K^{\alpha }L^{1-\alpha }.
\end{equation*}

\end{document}
